\documentclass{article}

%% PAQUETES

% Paquetes generales
\usepackage[margin=2cm, paperwidth=210mm, paperheight=297mm]{geometry}
\usepackage[spanish]{babel}
\usepackage[utf8]{inputenc}
\usepackage{gensymb}

% Paquetes para estilos
\usepackage{textcomp}
\usepackage{setspace}
\usepackage{colortbl}
\usepackage{color}
\usepackage{color}
\usepackage{upquote}
\usepackage{xcolor}
\usepackage{listings}
\usepackage{caption}
\usepackage[T1]{fontenc}
\usepackage[scaled]{beramono}

% Paquetes extras
\usepackage{amssymb}
\usepackage{float}
\usepackage{graphicx}
\usepackage{url}
\usepackage{courier}
\usepackage{enumitem} % For the enumerate list with correlative childs items

%% Fin PAQUETES


% Definición de preferencias para la impresión de código fuente.
%% Colores
\definecolor{gray99}{gray}{.99}
\definecolor{gray95}{gray}{.95}
\definecolor{gray75}{gray}{.75}
\definecolor{gray50}{gray}{.50}
\definecolor{keywords_blue}{rgb}{0.13,0.13,1}
\definecolor{comments_green}{rgb}{0,0.5,0}
\definecolor{strings_red}{rgb}{0.9,0,0}

%% Caja de código
\DeclareCaptionFont{white}{\color{white}}
\DeclareCaptionFont{style_labelfont}{\color{black}\textbf}
\DeclareCaptionFont{style_textfont}{\it\color{black}}
\DeclareCaptionFormat{listing}{\colorbox{gray95}{\parbox{16.78cm}{#1#2#3}}}
\captionsetup[lstlisting]{format=listing,labelfont=style_labelfont,textfont=style_textfont}

\lstset{
	aboveskip = {1.5\baselineskip},
	backgroundcolor = \color{gray99},
	basicstyle = \ttfamily\footnotesize,
	breakatwhitespace = true,   
	breaklines = true,
	captionpos = t,
	columns = fixed,
	commentstyle = \color{comments_green},
	escapeinside = {\%*}{*)}, 
	extendedchars = true,
	frame = lines,
	keywordstyle = \color{keywords_blue}\bfseries,
	language = Oz,                       
	numbers = left,
	numbersep = 5pt,
	numberstyle = \tiny\ttfamily\color{gray50},
	prebreak = \raisebox{0ex}[0ex][0ex]{\ensuremath{\hookleftarrow}},
	rulecolor = \color{gray75},
	showspaces = false,
	showstringspaces = false, 
	showtabs = false,
	stepnumber = 1,
	stringstyle = \color{strings_red},                                    
	tabsize = 2,
	title = \null, % Default value: title=\lstname
	upquote = true,                  
}

%% FIGURAS
\captionsetup[figure]{labelfont=bf,textfont=it}
%% TABLAS
\captionsetup[table]{labelfont=bf,textfont=it}

% COMANDOS

%% Titulo de las cajas de código
\renewcommand{\lstlistingname}{Código}
%% Titulo de las figuras
\renewcommand{\figurename}{Figura}
%% Titulo de las tablas
\renewcommand{\tablename}{Tabla}
%% Referencia a los códigos
\newcommand{\refcode}[1]{\textit{Código \ref{#1}}}
%% Referencia a las imagenes
\newcommand{\refimage}[1]{\textit{Imagen \ref{#1}}}


\begin{document}

% Inserción del título, autores y fecha.
\title{
	  \Large 75.44 Admin. y Control de Proy. Informáticos I \\ 
	  \medskip\large Facultad de Ingeniería, Universidad de Buenos Aires \\
	  \bigskip\Huge Informe: Ejercicio N° 0  \\
	  \medskip\huge\textit{``Sistema de Pickeo''} \\
	  \bigskip\bigskip\large\textbf{Federico Rossi (92.086), Ezequiel Reyes (92.119)} \\
	  \medskip\normalsize\texttt{\{federicomrossi, ezequielmreyes\}@gmail.com} \\
}
\date{}
\maketitle




%
% Enunciado
%
\section{Enunciado}
	
	Una empresa de venta directa \textit{Direcseller} tiene la siguiente configuración de sistemas:

	\begin{itemize}
		\itemsep=3pt \topsep=0pt \partopsep=0pt \parskip=0pt \parsep=0pt

		\item Un ERP World Class que cubre los módulos de Administración y Finanzas, Compras y Abastecimiento, Cuentas a Pagar;

		\item Un sistema ``Made In House'' hecho en Cobol que corre sobre Unix pero sobre servidores ya obsoletos que pronto deberán migrarse. Este sistema cubre los módulos de Facturación, Ventas y sobre todo, la carga de pedidos de la fuerza de calle (Revendedoras) y cuentas a cobrar;

		\item Una aplicación hecha en Java y mantenida por la misma empresa que cubre la parte de marketing y planificación de ventas.

	\end{itemize}
	\medskip

	La mayoría de los productos los importa pero gran parte de ellos los hace el país, aunque no tiene producción propia sino que los hace producir por terceros. En su planta del Gran Buenos Aires cuenta con una línea que se encarga de armar los pedidos de las revendedoras (Pickeo), la cual cuenta con una gran cinta donde van las cajas donde un grupo de operadores coloca en cada caja los productos del pedido. Ese sistema está soportado en Cobol.
	\par
	Recientemente a migrado exitosamente el ERP a la última versión del software liberada por el fabricante. Ese fue el paso previo al gran proyecto que le han pedido a usted que lidere, el cual consiste en:

	\begin{itemize}
		\itemsep=3pt \topsep=0pt \partopsep=0pt \parskip=0pt \parsep=0pt

		\item Migración del sistema Cobol al ERP, o sea, aparte de lo que ya esta en el ERP deberá quedar: Facturación, Ventas, cuentas a cobrar. Aunque la carga de pedidos que se hacía por escáner no es soportada por el ERP, por lo que debrá hacerse en otra aplicación e interfasearse;

		\item Interfaz del ERP con la aplicación Java;

		\item Migración Sistema de Pickeo al ERP, esta parte es la mas complicada ya que deberá contarse con escaners de pickeo de cajas fijos, escaners de productos, y programación de PLCs para los frenos de la línea. Las pickeadoras deberán contar con una pantalla lcd donde se indique el artículo a colocar dentro de la caja. Esto correra en Java. Pero los pedidos y los maestros de artículos estarán en el ERP. También cada caja que salga de la línea deberá ser facturada y la facturación se efectuará por el ERP.

	\end{itemize}
	\medskip

	Usted es empleado de la empresa Direcseller y participó en la migración anterior. Ahora se le pide que lidere este proyecto. Usted cuenta con la Consultora Codifica SA, que le brinda recursos funcionales y técnicos en el ERP y la empresa SoftAplicación que es candidata a tomar el desarrollo de Java para el Picking en el formato de llame en mano. El plazo que usted tiene para terminar este proyecto es de 1 año.

\bigskip




%
% Project Charter
%
\section{Project Charter}
	

\subsection{Justificación del proyecto}

	La migración  e integración de los distintos sistemas permitirá centralizar la gestión de los distintos módulos sectoriales que constituyen a la empresa. Esto reducirá la tasa de errores aumentando al mismo tiempo los niveles de productividad de los empleados. Además, ya no se tendrá un sistema obsoleto, sino que se contará con un sistema escalable y actualizable a futuro.
\medskip


\subsection{Objetivos del proyecto de alto nivel}

	Los objetivos se centran en migrar los módulos Facturación, Ventas y Cuentas a cobrar a ERP, migrar el sistema de Pickeo también a ERP, crear una aplicación que cumpla la función de interfacear entre el sistema ERP y la aplicación Java, la cual cubre la parte de marketing y planificación de ventas, y por último, crear una interfaz entre el módulo de pedidos y el sistema ERP. Todos el proyecto en general se realizará en el plazo de 1 año.
\medskip


\subsection{Requisitos de alto nivel}

	\begin{itemize}
		\itemsep=3pt \topsep=0pt \partopsep=0pt \parskip=0pt \parsep=0pt
		\item Migración completa del sistema de Cobol a ERP (tanto sistemas como servidores obsoletos).
		\item Interfaseo del ERP con la aplicación JAVA.
		\item Migración del sistema de Pickeo: interfaz para las pantallas en la cinta de producción en se realizará en JAVA, pedidos y gestión de artículos estarán en el ERP, al igual que la facturación de las cajas armadas.
	\end{itemize}

\medskip


\subsection{Criterios de aprobación del proyecto}

	Los criterios que se utilizarán parar evaluar el la aprobación del proyecto, se basarán de acuerdo a los siguientes puntos:
	\begin{itemize}
		\itemsep=3pt \topsep=0pt \partopsep=0pt \parskip=0pt \parsep=0pt
		\item Máxima cantidad de issues por día luego de los primeros 10 días de lanzamiento: 5.
		\item Obtener un sistema estable habiendo pasado los primeros 10 días de prueba: después de 20 días se espera obtener como máximo 15 issues.
		\item Soporte de pruebas de stress: servidor, sistema ERP, sistema de pickeo.
	\end{itemize}

\medskip


\subsection{Descripción del proyecto de alto nivel}

	El proyecto consiste principalmente en la migración de los sistemas realizados en lenguaje COBOL al sistema ERP recientemente liberado. Los circuitos gestionados por COBOL son los de Facturación, Ventas, Pedidos y Cuentas a cobrar; todos estos deben migrarse del sistema viejo al ERP (incluyendo la migración del servidor). Por otro lado debe poder interfasearse el ERP para poder compatibilizar los módulos de marketing y planificación de ventas realizados en JAVA. Por último debe realizarse la migración del Sistema de Pickeo al ERP, incluyendo la integración de escáneres para y programación PLC's para los frenos de la línea; esta migración incluye la interfaz gráfica realizada en JAVA para que se muestre la descripción del artículo a colocar dentro de la caja. 
Para realizar este proyecto se cuenta con un 1 año.

\medskip


\subsection{Riesgos de alto nivel}

	\begin{itemize}
	\itemsep=3pt \topsep=0pt \partopsep=0pt \parskip=0pt \parsep=0pt
	\item Migración del servidor obsoleto: es posible que para llevar a cabo esta migración deba prestarse atención en las incompatibilidades que pueda presentar el sistema realizado en COBOL. 
	\item Como los revendedores trabajan en la calle y acceden al sistema desde allí, debe prestarse atención a la seguridad del sistema en caso de robo, eso aumenta el alcance del proyecto ya que no hay especificaciones de seguridad realizadas en el sistema anterior.
	
	\end{itemize}
\medskip


\subsection{Limitaciones}

	\begin{itemize}
		\itemsep=3pt \topsep=0pt \partopsep=0pt \parskip=0pt \parsep=0pt
		\item Facturación electrónica: dependerá de la api provista por la AFIP.
		\item Escáneres: es posible que los escáneres deban ser importados, en ese caso se tiene que tomar en cuenta las probables retenciones de la aduana para la importación de los mismos. 
	\end{itemize}

\medskip


\subsection{Supuestos}

	\begin{itemize}
		\itemsep=3pt \topsep=0pt \partopsep=0pt \parskip=0pt \parsep=0pt
		\item Debido a que no hay especificaciones sobre versiones que se quieren utilizar para los distintos módulos, nosotros tomaremos esa decisión en base a lo que creamos mas conveniente para el mejor funcionamiento y estabilidad del sistema.
		\item Se cuenta con acceso total al sistema anterior, considerando bases de datos, aplicación, código fuente, etc.
		\item Se realizarán pruebas seguidas de integración por cada módulo del proyecto para garantizar correcto funcionamiento a medida que se va avanzando en el mismo.
		\item La logística para la interfaz de usuario necesaria correrá por cuenta de un diseñador externo. 
	\end{itemize}
\medskip


\subsection{Stakeholders}

	En el \textit{Cuadro 1} se detallaron dos de los stakeholders detectados en el presente caso. Se pueden encontrar más de estos, a saber, personal de los sectores afectados por las migraciones, personal de picking que deberá ser capacitada para utilizar la nueva maquinaria, entre otros.
	\medskip

	\begin{table}[!hbt]
		\begin{center}
		\begin{tabular}{| >{\raggedright}m{1.8cm} | >{\centering}m{1.5cm} | >{\raggedright}m{2cm} | >{\raggedright}m{1.5cm} | >{\centering}m{2.2cm} | >{\centering}m{1.8cm} | >{\centering}m{1.8cm} | >{\centering}m{2cm} |}
			\hline
			\rowcolor[gray]{0.9}\textbf{Nombre} & \textbf{Empresa, posición} & \textbf{Rol} & \textbf{Tipo} & \textbf{Necesidad de información} & \textbf{Intereses en el proyecto} & \textbf{Impacto en el proyecto} & \textbf{Información de Contacto} \tabularnewline
			\hline
			% A & B & C & D & E & F & G & H \tabularnewline
			% \hline
			Adrian Torrasco & Direcseller, Gerente de sector Facturación & Será consultado ante dudas del manejo de la Facturación y la integración del nuevo sistema en su sector. & Interno, receptor de información & Documen- tación para adaptar su sector al nuevo uso del sistema; Reuniones quincenales. & Constatar utilidad del nuevo sistema & Positivo, servirá de soporte en la migración & 4012-4587 int 746 \tabularnewline
			\hline
			Enrique Amado & SoftApli- cacion, CEO & Lider del proyecto de desarrollo en Java para el Picking en formato de llave en mano. & Externo, consultor & Requerimiento de personal interno que le otorge información necesaria para llevar a cabo el desarrollo. Semanal vía email con reuniones quincenales. & Sector de Picking & Positivo & ea@sa.com, 4891-3791 int 12 \tabularnewline
			\hline
		\end{tabular}
		\caption{Registro de Stakeholders.}
		\end{center}
	\end{table}


\bigskip


\newpage


%
% WBS
%
\section{WBS}


\subsection{Diagrama}


% Figura 1
\begin{figure}[h]
	\centering
	\includegraphics[width=1\textwidth]{images/7544-Ej00-WBS.png}
	\medskip
	\caption{Diagrama WBS para el proyecto.}
\end{figure}
\bigskip\bigskip\bigskip\bigskip


\subsection{Diccionario}

	A continuación se detalla el diccionario correspondiente al diagrama WBS de la \textit{Figura 1}:
	\bigskip

	\begin{enumerate}
		\itemsep=3pt \topsep=0pt \partopsep=0pt \parskip=0pt \parsep=0pt
		
		\item \textbf{Migración MIH a ERP}: Migración del sistema "Made in Home" , servidores obsoletos y sistema realizado en COBOL al nuevo en ERP. Cubre: Facturación, Ventas, Compras y Cuentas a pagar.

			\begin{enumerate}[label*=\arabic*.]

				\item \textbf{Facturación}: Módulo encargado de realizar las facturas con las ventas que llegan del módulo de ventas.

					\begin{enumerate}[label*=\arabic*.]
						\itemsep=3pt \topsep=0pt \partopsep=0pt \parskip=0pt \parsep=0pt

						\item \textbf{App}: Interfaz gráfica (formularios de facturación), generación automática de la factura electrónica y envío de la misma. 

						\item \textbf{Pruebas}: Pruebas de las 3 fases de facturación. Pruebas finales: 100 facturas creadas de forma concurrente.

						\item \textbf{Integración}: Integración de este módulo a la aplicación .

					\end{enumerate}
			
				\item \textbf{Ventas}: Módulo encargado de ingresar las ventas realizadas.

					\begin{enumerate}[label*=\arabic*.]
						\itemsep=3pt \topsep=0pt \partopsep=0pt \parskip=0pt \parsep=0pt

						\item \textbf{App}: Intefaz gráfica (formularios de venta, ingreso de productos), envío automático a facturación. 

						\item \textbf{Pruebas}: Test para la interfaz y la conexión con el módulo de facturación.

						\item \textbf{Integración}: Integración de este módulo a la aplicación.

					\end{enumerate}

				\item \textbf{Cuentas a cobrar}: Módulo encargado de gestionar las cuentas a cobrar a partir de las facturas que hay cargadas en el sistema.

					\begin{enumerate}[label*=\arabic*.]
						\itemsep=3pt \topsep=0pt \partopsep=0pt \parskip=0pt \parsep=0pt

						\item \textbf{App}: Interfaz gráfica (ABM con facturas y cuentas pendientes).

						\item \textbf{Pruebas}: Pruebas para la interfaz y conexión con el sistema para obtener las facturas y cuentas a cobrar.

						\item \textbf{Integración}:  Integración de este módulo a la aplicación .

					\end{enumerate}

			\end{enumerate}

		\item \textbf{App Pedidos}: Módulo encargado de gestionar los pedidos para los revendedoores.

			\begin{enumerate}[label*=\arabic*.]

				\item \textbf{App}: Gestionador de pedidos por parte de los revendedores.
			
				\item \textbf{Interfaz}: Formularios para generar los pedidos, ingreso de productos.

				\item \textbf{Pruebas}: Pruebas parciales para chequear el correcto funcionamiento del ingreso de los pedidos al sistema.

				\item \textbf{Integración}: Integración de este módulo a la aplicación.

				\item \textbf{Compras escaners}: Compra de los escáners necesarios para el sistema de pickeo.

			\end{enumerate}

		\item \textbf{Java}: Especificación de todas las tareas a desarrollar en JAVA.

			\begin{enumerate}[label*=\arabic*.]

				\item \textbf{Interfaz}: Interfaseo del sistema ERP. Interfaz gráfica para la descripción de los artículos en el sistema de pickeo.
			
				\item \textbf{Pruebas}: Test unitarios e integrales para la interfaz desarrollada.

				\item \textbf{Integración}: Integración de este módulo con el ERP.

			\end{enumerate}

		\item \textbf{Pickeo}: Especificación donde se especificará las tareas necesarias para la migración del sistema de Pickeo.

			\begin{enumerate}[label*=\arabic*.]

				\item \textbf{PLC}: Programación de los plc para los frenos de línea.

					\begin{enumerate}[label*=\arabic*.]
						\itemsep=3pt \topsep=0pt \partopsep=0pt \parskip=0pt \parsep=0pt

						\item \textbf{Programación}: migración desde Cobol al sistema ERP para los frenos de línea.

						\item \textbf{Prueba de frenos}: Test unitarios y de integración para los PLC.

					\end{enumerate}

				\item \textbf{Pickeadoras en Java}: Programación en JAVA para las pickeadoras.

					\begin{enumerate}[label*=\arabic*.]
						\itemsep=3pt \topsep=0pt \partopsep=0pt \parskip=0pt \parsep=0pt

						\item \textbf{App}: Aplicación para gestionar el proceso de Pickeo.

						\item \textbf{Pruebas}: Test unitarios y de integración para el proceso de Pickeo con muchos productos para estresar el sistema.

					\end{enumerate}

				\item \textbf{Integración ERP}: Tareas para integrar al sistema ERP los pedidos, los maestros, la facturación y las pruebas.

					\begin{enumerate}[label*=\arabic*.]
						\itemsep=3pt \topsep=0pt \partopsep=0pt \parskip=0pt \parsep=0pt

						\item \textbf{Pedidos}: Integración par atodos los pedidos de artículos al sistema de ERP

						\item \textbf{Maestros}: Integración para todos los maestros de artículos al sistema de ERP

						\item \textbf{Facturación}: Integración para todas las facturaciones de artículos al sistema de ERP

						\item \textbf{Pruebas}: Test integrales para poder chequear el buen funcionamiento del sistema ERP con los 3 módulos integrados.

					\end{enumerate}
			
				\item \textbf{Compras}: Especificación de la compra de materiales necesaria para llevar adelante el proyecto.

					\begin{enumerate}[label*=\arabic*.]
						\itemsep=3pt \topsep=0pt \partopsep=0pt \parskip=0pt \parsep=0pt

						\item \textbf{Escaners de Pickeo}: Compra de escaners de pickeo para migrar el sistema de pickeos.

						\item \textbf{Escaners de productos}: Compra de escaners de productos para la identificación de los productos, para poder subirlos automáticamente al sistema nuevo. 

						\item \textbf{Equipos PLC}: Compra de equipos PLC para los frenos de la línea.

					\end{enumerate}

			\end{enumerate}

		\item \textbf{Administración del proyecto}: Especificación de las tareas necesarias referentes a toda la administración del proyecto, a la asignación de los recursos.

			\begin{enumerate}[label*=\arabic*.]

				\item \textbf{Recursos}: Especificación de los recursos necesarios, divididos en técnicos y funcionales.

					\begin{enumerate}[label*=\arabic*.]
						\itemsep=3pt \topsep=0pt \partopsep=0pt \parskip=0pt \parsep=0pt

						\item \textbf{Técnicos}: Desarrolladores, analistas, tester, diseñadores.

						\item \textbf{Funcionales}: [ Colocar texto aquí ]

					\end{enumerate}
			
				\item \textbf{Tercerización de Picking}: [ Colocar texto aquí ]

				\item \textbf{Estimación de Tiempos}: Análisis y administración de los tiempos que tomará cada entregable de este proyecto.
			\end{enumerate}

	\end{enumerate}
	\medskip

\bigskip



% % CONSIDERACIONES DE DISEÑO
% \section{Consideraciones de diseño}

% 	Para la correcta implementación de la solución fue necesario plantear y establecer cómo se debería comportar el sistema ante ciertas situaciones que no fueron especificadas en el enunciado del problema. A continuación se listan las contemplaciones instauradas:

% \begin{itemize}
% 	\itemsep=3pt \topsep=0pt \partopsep=0pt \parskip=0pt \parsep=0pt

% 	\item Al producirse una solicitud de trabajo por parte de un cliente, si el servidor le indica a este último que no hay trabajo para asignarle, entonces el cliente cerrará la conexión que los vinculaba y dará por finalizada su ejecución;

% 	\item El servidor almacenará todas las posibles claves enviadas por los clientes, de manera de que, una vez se indique la detención del mismo, se pueda determinar si existe una ambig\"uedad o si es única la clave;

% 	\item En el envío y en la recepción de datos se considera como fin de mensaje al caractér ``\textit{$\backslash$n}'' establecido por el protocolo.

% 	\item El programa del lado servidor no finalizará su ejecución a menos que se ingrese ``\textit{q$\backslash$n}'' por la entrada estándar.
% \end{itemize}
% \medskip




% % DISEÑO
% \section{Diseño}

% 	Como se mencionó anteriormente, se optará por una solución distribuida siguiendo un esquema cliente-servidor. De esta manera, poseeremos dos implementaciones independientes una de la otra, pero que trabajarán en conjunto una vez establecida la conexión entre las mismas. Por lo tanto, vamos a tener un programa \textit{servidor} y otro programa \textit{cliente}.
% 	\par
% 	En los apartados que siguen pondremos la atención en aquellos aspectos de la implementación de ambos programas que pueden ser relevantes a causa de su complejidad o particularidad. En estos se describen los inconvenientes que presentan y la forma en que fueron resueltos.
% \bigskip



% % DISEÑO - Cliente
% \subsection{Cliente}
	
% 	Sin duda alguna, este ente del esquema es el menos complejo ya que su tarea se reduce a establecer inicialmente una conexión con el servidor, para luego solicitarle una parte del trabajo a realizar.
% 	\par
% 	En el caso de que el servidor le indique que no hay trabajo disponible, el cliente simplemente finalizará su ejecución. En la situación opuesta, recibirá un mensaje encriptado (codificado por el servidor en hexadecimal para su envío a través del socket) y el rango de claves, siendo su tarea probar cada una de ellas por fuerza bruta, notificando a través del socket cuales son posibles claves.
% 	\par
% 	Finalizadas las pruebas terminará su ejecución automáticamente dando fin a la conexión con el servidor.
% 	\par
% 	Debe tenerse en cuenta que ante cualquier problema surgido en la conexión, ya sea por una interrupción en la misma o por la llegada de mensajes erróneos que no se ajustan al protocolo u orden esperado, el programa cliente también finalizará su ejecución.
% 	\par
% 	Profundizando medianamente, este ente es implementado por la clase \textit{Cliente}, la cual, por las razones expuestas al inicio del apartado, se consideró que no sea un hilo independiente del programa invocante.
% 	\bigskip



% % DISEÑO - Cliente - Pruebas sobre el código Draka
% \subsubsection{Pruebas sobre el \textit{Código Draka}}

% 	Desde un principio se ha establecido que el cliente es el encargado de probar el rango de claves, asignado por el servidor, en el código utilizado por los Draka para encriptar y desencriptar sus mensajes.
% 	\par
% 	Para lograr una solución más elegante y mas limpia, se tomó la decisión de encapsular dicho código en una clase de métodos estáticos, \textit{CodigoDraga}, la que a su vez proporciona un método que recibe una clave y que se encarga de realizar la prueba sobre el código Draka, devolviendo como resultado de esta acción un valor lógico, es decir, si retorna ``true'' significará que la clave pasó la prueba y efectivamente puede ser considerada una posible clave, o si devuelve ``false'' la clave probada deberá ser descartada.
% 	\bigskip\bigskip


% % DISEÑO - Servidor
% \subsection{Servidor}

% 	Para el funcionamiento del servidor se ha propuesto utilizar una lógica muy simple. Esta consiste en tener a la clase \textit{Servidor}, la cual es un hilo de ejecución independiente del hilo principal del main, esperando por solicitudes de conexión por parte de los clientes. Cuando se recibe una petición de conexión por parte de un cliente, el mismo servidor se encarga de crear un objeto de la clase \textit{ConexionCliente}. A este objeto se le pasa el file descriptor correspondiente al nuevo socket que mantendrá en contacto al cliente con este último objeto. Además, el objeto que representa a la conexión será almacenado en una lista de conexiones existentes, para permitir al servidor saber que conexiones debe cerrar cuando llegue el momento de concluir la ejecución.
% 	\par
% 	Como se puede notar, esta clase, considerada la principal del lado servidor, simplemente se limita a la escucha a través de un socket, el cual solo es utilizado con este único fin. Todas las demás tareas de comunicación son derivadas y delegadas a las demás entidades, de las que haremos mención en los siguientes apartados.
% 	\bigskip\smallskip



% % DISEÑO - Servidor - Conexion con el cliente
% \subsubsection{Conexión con el cliente}

% 	Anteriormente se adelantó que la clase \textit{Servidor}, ante una nueva petición de conexión por parte de un host, creaba un objeto del tipo \textit{ConexionCliente}. Esto significa que este objeto es de ahora en más el único responsable del enlace con el cliente que realizó la petición, es decir, es el único encargado de mantener la comunicación con su par del otro lado del socket. Entre estos se hará el envío de mensajes protocolares los cuales permitirán la asignación de tareas, notificación de posibles claves, etc. Para esto, el ente que administra la conexión se mantendrá en estrecha relación con el \textit{controlador de tareas} del servidor.
% 	\bigskip\smallskip



% % DISEÑO - Servidor - Control de tareas
% \subsubsection{Control de tareas}

% 	Otro de los entes que se encuentra a las ordenes del servidor, ya que es creado por este mismo y no puede existir mas allá del alcance de este, es el representado por la clase \textit{ControladorDeTareas}. Como su nombre lo indica, es el encargado de administrar las tareas. Más especificamente, su labor está centrada en:
% 	\medskip

% 	\begin{itemize}
% 		\itemsep=3pt \topsep=0pt \partopsep=0pt \parskip=5pt \parsep=0pt

% 		\item \textit{División de tareas}: deberá fraccionar las partes de trabajo que deben ser asignadas a cada cliente, como así también indicar cuando ya no hay mas tareas por repartir;

% 		\item \textit{Recepción de claves}: cada objeto del tipo \textit{ConexionCliente}, al recibir una clave de parte del cliente notificará al controlador de tareas despachándole esta misma clave;

% 		\item \textit{Finalización de tarea}: cuando un cliente envía el mensaje de que ha concluido con su parte del trabajo, el objeto \textit{ConexionCliente} asociado a dicha conexión notificará al controlador de tareas de tal evento;

% 		\item \textit{Notificar estado de tareas}: deberá informarle al servidor del estado de las tareas cuando este lo solicite.

% 	\end{itemize}
% 	\bigskip


% 	De estas cuatro labores debemos destacar una en particular: la \textit{recepción de claves}. Esto se debe a que presenta una de las problemáticas mas interesantes. 
% 	\par
% 	Tal como se mencionó, el controlador de tareas recibirá las claves, pero hay que tener en cuenta que cada objeto \textit{ConexionCliente} se encuentra corriendo en un hilo independiente y que se puede dar el caso de que dos de estos hilos traten de notificar al mismo tiempo la llegada de una nueva clave. Es aquí donde se decidió convertir a la clase \textit{ControladorDeTareas} en thread-safe, es decir, con soporte para multithreading. Por lo tanto, se incluyó una auto protección (un objeto de tipo \textit{Mutex}) en la clase, lo cual evita que las instancias de esta misma sean accedidas al mismo tiempo por hilos diferentes.
% 	\par
% 	Por otro lado, al ser informado de la llegada de una nueva clave, el controlador de tareas debe tomar a esta última e ingresarla en una lista proporcionada por la clase \textit{Servidor}. Esta lista presenta problemas semejantes con respecto al multithreading, razón por la cual será discutida en el apartado que sigue, dándose detalles de cómo es que se evitan los conflictos de acceso a un objeto por parte de múltiples hilos.
% 	\bigskip\smallskip



% % DISEÑO - Servidor - Clase Lista
% \subsubsection{Clase \textit{Lista}}
	
% 	Como ya es sabido, la \textit{STL} \cite{STL} no proporciona estructuras con soporte para multithreading. Este es el caso del contenedor \textit{list}, el cual no es \textit{thread-safe}. Ante la necesidad de la utilización de esta en el lado del servidor, de manera de poder albergar a las claves que arriven desde los clientes (como se explicó en el apartado anterior), es que se decidió extender a \textit{list} y agregarle el debido soporte.
% 	\par
% 	Para esto, hemos creado una clase \textit{Lista}, la cual utiliza internamente al contenedor \textit{list}, pero con la salvedad de que cada método se encuentra protegido de los múltiples accesos posibles desde distintos threads. 
% 	\par
% 	Es así que se estableció como un atributo interno y privado un objeto de tipo \textit{Mutex}, el cual es bloqueado en cada invocación a los métodos, evitando que mientras se ejecutan las instrucciónes de este último, otros threads puedan hacer uso de la lista, quedando en su defecto a la espera de la liberación del mutex.
% 	\bigskip\medskip




% % PROTOCOLO Y COMUNICACION
% \subsection{Protocolo y comunicación}

% 	Respecto a la forma en que se comunican los clientes y el servidor, ya se adelantó anteriormente la utilización de un protocolo fijo e irremplazable. Para poder unificarlo y que los objetos comprendan los mismos tipos de mensajes se creó la clase \textit{Protocolo}, la cual simplemente alberga constantes que definen los distintos tipos de avisos a intercambiar entre los entes.
% 	\par
% 	Por otro lado, para abstraer la forma de enviar y recibir datos, se decidió establecer como intermediario a la clase \textit{Comunicador}, la cual se encarga de emitir o de recibir datos (a través del socket proporcionado por su empleador) teniendo en cuenta el protocolo antes mencionado.
% 	\bigskip\bigskip




% % FUTURAS MEJORAS
% \section{Futuras mejoras}

% 	Se listan aquí posibles mejoras a realizar en el futuro, como así también, ciertas falencias que fueron descubiertas habiéndose hecho ya la entrega del producto, a fin de denotar conciencia sobre el trabajo realizado:

% 	\begin{itemize}
% 		\itemsep=3pt \topsep=0pt \partopsep=0pt \parskip=0pt \parsep=0pt

% 		\item Mejorar informe de errores en clase \textit{Cliente}, de manera de proporcionar información mas detallada acerca de la desconexión del socket;

% 		\item Clases \textit{Thread} y \textit{Mutex} mas completas con respecto al soporte de mas opciones;

% 		\item Mejorar empaquetamiento de la clase \textit{Comunicador} y el armado de mensajes por parte de este.

% 	\end{itemize}
% 	\bigskip
% \newpage




% % ESQUEMA DEL DISEÑO
% \section{Esquema del diseño}

% 	A continuación, en la \textit{Figura 1}, se ilustra el diagrama de clases principal, donde se pueden ver las entidades que intervienen tanto en el
% 	servidor como así también en el cliente. Cabe destacar que se han omitido,
% 	para mayor comprensión, ciertas clases secundarias que hacen a la totalidad del sistema, pero que su presencia no es indispensables para el entendimiento del funcionamiento general del mismo. 
% 	\bigskip\bigskip\bigskip\bigskip


% % Figura 1
% \begin{figure}[h]
% 	\centering
% 	\includegraphics[width=0.9465\textwidth]{images/diagrama_p1.png}
% 	\medskip
% 	\caption{Diagrama de clases principal del cliente-servidor.}
% \end{figure}
% \bigskip\bigskip\bigskip\bigskip

	
% 	Por último, en la \textit{Figura 2}, se muestra un diagrama de clases ampliado del lado servidor. En este se ilustra qué clases son un \textit{Thread}, es decir, que clases proveen objetos que corren en distintos hilos al ser instanciadas. Además se puede observar la relación entre entidades, las cuales fueron descriptas en secciones anteriores.

% \newpage
% % Figura 2
% \begin{figure}[h]
% 	\centering
% 	\includegraphics[width=0.8\textwidth]{images/diagrama_p2.png}
% 	\medskip
% 	\caption{Diagrama de clases del servidor.}
% \end{figure}
% \bigskip\bigskip\bigskip\bigskip




% % REFERENCIAS
% \begin{thebibliography}{99}

% 	\bibitem{ASCII} Código ASCII, \url{http://en.wikipedia.org/wiki/ASCII}
% 	\bibitem{FB} Ataque por Fuerza Bruta (Brute-force attack), \url{http://en.wikipedia.orig/wiki/Brute-force_attack}
% 	\bibitem{STL} Standard Template Library, \url{http://en.wikipedia.org/wiki/Standard_Template_Library}
% 	\end{thebibliography}






\end{document}
